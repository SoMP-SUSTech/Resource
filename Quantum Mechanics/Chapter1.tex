\section{Introduction}
Quantum mechanics is the most elegant theory that human beings have ever built. It helped us to create computers (by which I mean, classical computers), to invent new materials, or even to come up with new ideas in mathematics. The success of it is so fruitful that until now, a hundred years later, we are still making new discoveries out of it that are beyond our imagination.

Books being written, stories being told, heroes being talked about all the time. And now modern science fictions especially favour that magic word -- QUANTUM, the word that injects magic power into every seemingly trivial noun that follows. The brightness of quantum mechanics  seems to be shining enough so that it won't be harmful to ignore the dark.

However, it is not elegant enough.

If you ever come close, you shall find imperfections hiding in every corner of the theory. You could also do fairly well away from the corners. Nevertheless, those annoying dirty corners are where the astonishing secrets, if there were, prefer to live.

In a seminar lasting for less than half a year, I have no dare to say I could tell the whole story, with most of which still being a riddle for myself. Thus we shall focus only on a part of it. Then let's decide which. Since there have already been tons of books about quantum mechanics and since the seminar is for mathematicians, we should tell something different.

Instead of the calculations, we shall focus more on structures; instead of analysis, we shall focus more on algebra; and instead of the success, we shall focus more on the failure.

\subsection{Background}
Quantum mechanics was not noticed until 1905, when Planck entered the stage. But actually many optical phenomena observed before that can only be explained with quantum mechanics. What happened then has been talked about in so many books so I won't do it here.

One thing to note is that in many books quantum mechanics is treated as some opposite of the classical one. However, that treatment is not totally correct. Quantum mechanics shares a strong connection with the classical one, which is a fact but perhaps not a good one. I would like to say that maybe our quantum mechanics is not "quantum" enough so that it relies heavily on the framework of classical mechanics.

It is often said that quantum mechanics is really abstract. Maybe it is true. But a more interesting question to ask is "why isn't classical mechanics abstract?" There are three reasons for that. First of all, I think both of them are abstract when you are unfamiliar with them. Our well-established education system made us acquire lots of knowledge of Newtonian mechanics in high schools. After that, the world seems to work harmonically. Another reason is that classical mechanics is enough for most phenomena in every-day life, while it needs extreme experiment condition to see quantum effects. The last reason is that the word \textit{force} lies in the vocabulary far before classical mechanics being built. Actually the word \textit{force} can be regarded to have been redefined when we assign to it a magnitude. But in quantum mechanics, we created words and their quantitative descriptions at the same time.

However, for mathematicians, being abstract is of course never terrifying. So it won't be hard for us to proceed.

\subsection{Why is quantum mechanics so messy}
One amazing thing of quantum mechanics is that it was not built by one particular person, unlike Einstein's relativity, but by so many heroes of that time, which, however, also leads to the messiness of the theory.

\begin{aquote}{Rene Descartes}
"... that there is very often less perfection in works composed of several portions, and carried out by the hands of various masters, than in those on which one individual alone has worked. Thus we see that buildings planned and carried out by one architect alone are usually more beautiful and better proportioned than those which many have tried to put in order and improve, making use of the old walls which were built with other ends in view..."\cite{Discourse on Method}
\end{aquote}

As noticed by Descartes, it is not easy for distinct masters to tell the same story and make it coherent at the same time. After decades of struggling, now we finally arrived at a fairly coherent theory but with fundamental problems left unsolved, especially problems related to measurements in quantum mechanics.

There are around eight or nine different ways to describe quantum mechanics, with each of them being distinct from the others. Here I could only talk about the most well-established ones. It is unbelievable that those seemingly uncorrelated descriptions are telling the same story while all being correct.

Here I shall note the importance of equivalent descriptions for the same theory in physics. The most famous example, in my opinion, is classical mechanics. There are three equivalent descriptions, i.e., the Newtonian, the Lagrangian and the Hamiltonian. If a complete coherent theory is all we want, maybe the Newtonian mechanics is enough, why bothering looking for its equivalence?

However, quantum mechanics inherits the language of Hamiltonian mechanics, which itself inherits from the Lagrangian one. One interesting question to think is that if we did not have the Lagrangian mechanics nor the Hamiltonian mechanics, how should we build the quantum theory? Above all, I could claim that people would still have discovered quantum mechanics, but maybe with a totally different language as we are using today, since no matter what theory we have, the experiments were showing that the classical theory was incomplete. One fatal problem for Newtonian mechanics is that the Newtonian mechanics relies on the concept of force, which is hard to find a corresponding in quantum mechanics, while the other two description of classical mechanics do not.

So even though maybe we could use some technique to "quantize" the Newtonian mechanics, its brothers are more nature choices.

Ergo, the importance of equivalent descriptions is that they provide different languages and having more languages provides better chance to describe the unknown.

Now we should also be positive seeking more equivalent languages for quantum mechanics. Maybe the one you find would be the key to the next page.

\subsection{Different mathematical approaches to QM}
Let's talk about maths. As advertised, there are many different mathematical approaches to quantum mechanics. Among them there are mainly two types, one being analysis and the other being algebra.

The most important formula in quantum mechanics is the well-known Schrodinger equation. So we could use methods in PDE to study quantum mechanics, focusing on solving the Schrodinger equation. Maybe like Griffiths' choice, to put the Schrodinger equation in the very first page of the book as a friendly opening to the lovely freshmen planning to major in physics. And then talk about separations of variables, perturbations, etc. That a good way to organize quantum mechanics, as it taught you how to calculate. Or maybe being more mathematical, giving lots of tricky boundary conditions to see how the poor particle reacts.

Another approach will be algebraic, mainly linear algebra. I think this approach is more natural than the former one.

In between lies the functional analysis. Actually this is the most natural way. But if we choose to ignore the tricky problems caused by infinite dimensions, it reduces to linear algebra. That is why linear algebra is a good approach. I must admit there is a huge gap here by ignoring infinity but since there are already fruitful results doing so, physicists cannot help to touch them. Also, the operator algebra part of functional analysis is what is really useful in quantum mechanics and that is quite algebraic so quantum mechanics can be really algebraic.

Before we get started, let's first do a brief review of the necessary knowledge in classical mechanics.