\documentclass{article}
\title{Category Theory and Abstract Algebra}
\author{Li, Yunsheng}
\date{}
\usepackage{amsmath}
\usepackage{amsfonts}
\usepackage{underscore}
\usepackage{amssymb}
\usepackage{mhchem}
\usepackage{geometry}
\usepackage{mathtools}
\usepackage{stmaryrd}
\usepackage{extarrows}
\usepackage{bm}
\usepackage{amsthm}
\usepackage{tikz-cd}
\usepackage{enumitem}\usepackage[colorlinks,linkcolor=black]{hyperref}
\usepackage{titlesec}

\newtheorem{theorem}{Theorem}[section]
\newtheorem{corollary}{Corollary}[theorem]
\newtheorem{proposition}{Proposition}[section]
\newtheorem{lemma}[theorem]{Lemma}

\theoremstyle{definition}
\newtheorem{definition}{Definition}[section]

\theoremstyle{definition}
\newtheorem{example}{Example}[section]

\theoremstyle{remark}
\newtheorem*{remark}{Remark}

\pagestyle{plain}

\DeclareMathOperator{\Ima}{Im}
\DeclareMathOperator{\Obj}{Obj}
\DeclareMathOperator{\Mor}{Mor}
\DeclareMathOperator{\dom}{dom}
\DeclareMathOperator{\cod}{cod}
\DeclareMathOperator{\Hom}{Hom}
\DeclareMathOperator{\End}{End}
\DeclareMathOperator{\Aut}{Aut}
\DeclareMathOperator{\Sym}{Sym}
\DeclareMathOperator{\coker}{coker}

\geometry{a4paper,left=1.5cm,right=1.5cm,top=1.5cm,bottom=1.5cm}

\begin{document}
\maketitle
\tableofcontents

\section*{Introduction}
This thesis will give an introduction to basic category theory, and show how category theory might be related or applied to abstract algebra by several examples and propositions. Also, we would mention some notions about abstract algebra which aren't mentioned in the course, and some notions in topology theory. \par
We would admit facts in set theory and assume the axiom of choice. We would make definitions and notations as detailed as possible, but omit proofs for simple facts, give references or proofs with details omitted for complicated theorems like the Yoneda Lemma. The main references of this thesis are Riehl, Emily. \textsl{Category Theory in Context}, Aluffi, Paolo. \textsl{Algebra: Chapter 0} and Munkres, James. \textsl{Topology}.
\newpage
\section{Category}
In this section we will introduce some elementary notions in category theory and give several examples of categories.
	
	\begin{definition}[Category]
	A \textbf{category} $\mathcal{C}$ consits of:
		\begin{itemize}
			\item a collection of \textbf{objects}. We would denote this collection by $\Obj(\mathcal{C})$. If something $X$ is an object of $C$, then we say $X$ is in $\Obj(\mathcal{C})$, which is denoted by $X\in\Obj(\mathcal{C})$.
			\item a collection of \textbf{morphisms}. We would denote this collection by $\Mor(\mathcal{C})$. Similarly, if something $f$ is a morphism of $\mathcal{C}$, then we say $f$ is in $\Mor(\mathcal{C})$, denoted by $f\in \Mor(\mathcal{C})$.\footnote{Although we used the notation $\in$ as is in set theory, neither $\Obj(\mathcal{C})$ nor $\Mor(\mathcal{C})$ was asked to be as samll as a set. As we shall see, for the category $\mathrm{Set}$, neither $\Obj(\mathrm{Set})$ nor $\Mor(\mathrm{Set})$ is a set.}
		\end{itemize}
	where the morphisms satisfy the following properties (or, say, have the following information):
		\begin{itemize}
			\item Each morphism of $\mathcal{C}$ specifies two objects of $\mathcal{C}$ called its \textbf{domain} and \textbf{codomain}, respectively. The notation $f:X\to Y$, read ``$f$ is a morphism from $X$ to $Y$'', signifies that $f$ is a morphism with domain $X$ and codomain $Y$. With this notation, a morphism is also called an \textbf{arrow} pointing from its domain to its codomain. Given a morphism $f$, we denote its domain by $\dom f$ and its codomain by $\cod f$. Two morphisms $f,g$ with $\dom f=\dom g$ and $\cod f=\cod g$ are said to be \textbf{parallel}, and the notation $f,g:X\rightrightarrows Y$ signifies that $f$ and $g$ are parallel morphisms with domain $X$ and codomain $Y$. Given two objects $X$ and $Y$ of $\mathcal{C}$, we denote the collection of all morphisms of $\mathcal{C}$ with $X$ as its domain and $Y$ as its codomain by $\Hom_\mathcal{C}(X,Y)$.\footnote{Again, $\Hom_\mathcal{C}(X,Y)$ might not be a set, though we would let the notation $f\in\Hom_\mathcal{C}(X,Y)$ signify that $f$ is a morphism of $\mathcal{C}$ with $\dom f=X$ and $\cod f=Y$.} A morphism $f$ with $\dom f=\cod f$ is called an \textbf{endomorphism}, and we denote the collection $\Hom_\mathcal{C}(X,X)$ by $\End_\mathcal{C}(X)$. The subscript $\mathcal{C}$ could be omited, if there comes no confusion about that which category we are focused on.
			\item For any two morphisms $f,g$ of $\mathcal{C}$ with $\cod f=\dom g$, there exists a morphism $h$ of $\mathcal{C}$ with $\dom h=\dom f$ and $\cod h=\cod g$ specified by $f$ and $g$ according to some rules called the \textbf{composition law} in $\mathcal{C}$. We say that $h$ is the \textbf{composite morphism} of $g$ composed with $f$ (according to the composition law in $\mathcal{C}$), usually denoting it by $gf$ instead of $h$.\footnote{Sometimes we might want to specify that according to composition law in which category is the composite morphism determined, in which case we would put some notation between $g$ and $f$. For example, $g\circ_\mathcal{C} f$ might denote that $g\circ_\mathcal{C} f$ is the composition morphism of $g$ composed with $f$ according to the composition law in the category $\mathcal{C}$.} For morphisms $f$ and $g$ satisfying the condition $\cod f=\dom g$, we say that they are \textbf{composable}. We may express composition by notations below:
				\[f:X\to Y,\ \ g:Y\to Z\ \ \ \leadsto\ \ \ \ gf:X\to Z.\]
			\item For any composable triple of morphisms $f,g,h$ of $\mathcal{C}$, we have
				\[h(gf)=(hg)f.\]
			That is, the composition law is associative. Therefore, we can omit the parameter and write $hgf$ instead. Namely,
				\[f:X\to Y,\ \ g:Y\to Z,\ \ h:Z\to W\ \ \ \ \leadsto\ \ \ \ hgf:X\to W.\]
			\item For any object $X$ of $\mathcal{C}$, there exists an \textbf{identity morphism} $1_X:X\to X$, which satisfies that
			\[1_Xf=f,\ \ g1_X=g\]
			for any $f,g\in \Mor(\mathcal{C})$ s.t. $\cod f=\dom g=X$. It is immediate that the identity morphism of an object in a certain category is unique. We would always denote the identity morphism of an object $X$ by $1_X$, if there comes no confusion about that of which category the identity morphism is. A category with only identity morphisms is said to be \textbf{discrete}.
		\end{itemize}
	\end{definition}
We state several basic notions about a category before we give examples of categories.
	\begin{definition}[Subcategory]
	Let $\mathcal{C},\mathcal{D}$ be two categories, then $\mathcal{C}$ is said to be the \textbf{subcategory} of $\mathcal{D}$, if the composition law in $\mathcal{C}$ and $\mathcal{D}$ coincidents and every element in $\Obj(\mathcal{C})$ is in $\Obj(\mathcal{D})$, every element in $\Mor(\mathcal{C})$ is in $\Mor(\mathcal{D})$. We might denote the last two conditions by $\Obj(\mathcal{C})\subset\Obj(\mathcal{D})$ and $\Mor(\mathcal{C})\subset\Mor(\mathcal{D})$ for conveniont.
	\end{definition}

	\begin{definition}[Small, Locally Small]
	A category $\mathcal{C}$ is \textbf{small} if $\Mor(\mathcal{C})$ is a set, or equivalently, there is a bijection from $\Mor(\mathcal{C})$ to some set. $\mathcal{C}$ is \textbf{locally small} if for any $X,Y\in \Obj(C)$, $\Hom_\mathcal{C}(X,Y)$ is a set.
	\end{definition}
There are categories that are neither small nor locally small. This thesis would mainly focus on locally small categories. For a small category $\mathcal{C}$, $\Obj(\mathcal{C})$ is also a set since there is an injection\footnote{Although what we said as a ``collection'' might be too large to be a set, we can still define notions like maps, inclusion, injective, surjective, ..., just as how we defined them in set theory.} $\Obj(\mathcal{C})\to\Mor(\mathcal{C}):X\mapsto 1_X$.
	\begin{definition}[Initial, Terminal]
	Given a category $\mathcal{C}$. An object $A$ of $\mathcal{C}$ is an \textbf{initial object} of $\mathcal{C}$ if $\Hom_\mathcal{C}(A,X)$ contains only one element for any $X\in \Obj(\mathcal{C})$. We might say such an object is \textbf{initial} instead of saying it is an initial object of some category, if there comes no confusion about that which category we are focused on. \\
	Dually, an object $A$ of $\mathcal{C}$ is a \textbf{terminal object} of $\mathcal{C}$ if $\Hom_\mathcal{C}(X,A)$ contains only one element for any $X\in \Obj(\mathcal{C})$. Similarly, we might say an object is \textbf{terminal} instead of saying it is a terminal object of some category.
	\end{definition}
In particular, for an object $A$ which is either initial or terminal, $\End(A)$ contains only one element, the identity map $1_A$. It's possible for an object to both initial and terminal, as we shall see in the category $\mathrm{Grp}$.
	\begin{definition}[Isomorphism]
	Given a category $\mathcal{C}$. A morphism $f:X\to Y$ of $\mathcal{C}$ is an \textbf{isomorphism} if there exists a morphism $g\in\Hom_\mathcal{C}(Y,X)$ s.t.
	\[gf=1_{X},\ \ fg=1_{Y}.\]
	Such a $g$ is called the \textbf{inverse} of $f$. If there is an isomorphism between $X$ and $Y$, then $X$ and $Y$ are said to be \textbf{isomorphic}, denoted by $X\cong Y$.
	\end{definition}
It's clear that isomorphic is an equivalence relation, hence we used the word ``between'' here. Also, the inverse of an isomorphism is unique, hence the morphism $g$ mentioned above might be denoted as $f^{-1}$.\par
An immediate fact is that, given a category $\mathcal{C}$, then all initial objects of $\mathcal{C}$ are isomorphic to each other, so do all terminal objects of $\mathcal{C}$.
	\begin{definition}[Groupoid]
	A \textbf{groupoid} is a category whose morphisms are all isomorphisms.
	\end{definition}
	Now we can redefine what we called a ``group'' in abstract algebra in a categorical view:
	\begin{definition}[Group]
	A \textbf{group} is a locally small groupoid who has only one object.
	\end{definition}
	In fact, the groupoid mentioned above is small. In this point of view, given a group under our traditional notion, it induces a category, namely a locally small groupoid who has only one object. 
	\begin{example}
	 We can now state several examples of categories below:
	\begin{enumerate}[label=(\roman*)]
	\item Given a group $G$, then it induces a category, denoted by $\mathrm{B}G$, in which:
		\begin{itemize}
			\item Objects: an abstract-nonsense point $\bullet$.
			\item Morphisms: Elements in $G$, with $\bullet$ being their domains and codomains.
			\item Composition law: The composition in the group $G$. That is, if we let $\circ$ denotes the group operation, then for any $f,g\in G(=\Mor(\mathrm{B}G))$,
			\[gf\coloneqq g\circ f.\]
		\end{itemize}
	The identity morphism $1_\bullet$ is exactly the identity in $G$, and every morphism in $\mathrm{B}G$ is an isomorphism, with its inverse in $G$ being its inverse as a morphism.
	\item The category of sets, denoted by $\mathrm{Set}$, in which:
		\begin{itemize}
			\item Objects: All sets.
			\item Morphism: All set-functions, with their domain  and codomain the same as the domain and codomain when seen as set-functions.\footnote{When we define a category, if the morphism is chosen to be something with pre-defined domains and codomains, we shall omit the words ``with their domain  and codomain the same as the domain and codomain when seen as set-functions'' and take this as default.}
			\item Composition law: Composition of set-functions.
		\end{itemize}
	The identity morphism for each object (set) is the identity map from the set to itself. A morphism of $\mathrm{Set}$ is an isomorphism if and only if it is a bijection. The initial object in $\mathrm{Set}$ is the empty set $\varnothing$, and the terminal object is the singleton $\{*\}$. $\mathrm{Set}$ is locally small but not small.
	\item The category of groups, denoted by $\mathrm{Grp}$, in which:
		\begin{itemize}
			\item Objects: All groups.
			\item Morphisms: All group homomorphisms.
			\item Composition law: Composition of set-functions.
		\end{itemize}
	The identity morphism is the identity group homomorphism. A morphisms of $\mathrm{Grp}$ is an isomorphism if and only if it is a group isomorphism. The initial object and the terminal object in $\mathrm{Grp}$ are both the trivial group $\{*\}$, namely the trivial group is both initial and terminal in $\mathrm{Grp}$. Again, $\mathrm{Grp}$ is locally small but not small, for there is an injection from $\Obj(\mathrm{Set})$ to $\Obj(\mathrm{Grp})$, that sends each set to the free group on it; we shall define what a free group is after we introduce the notion of universal property.
\newpage
	\item The category of abelian groups, denoted by $\mathrm{Ab}$, in which:
		\begin{itemize}
			\item Objects: All abelian groups.
			\item Morphisms: All group homomorphisms between abelian groups.
			\item Composition law: Composition of set-functions.
		\end{itemize}
	$\mathrm{Ab}$ is a subcategory of $\mathrm{Grp}$. $\mathrm{Ab}$ is also locally small but not small, for there is an injection from $\Obj(\mathrm{Set})$ to $\Obj(\mathrm{Ab})$ that sends each set to the free abelian group on it; we shall define what a free abelian group is after the notion of universal property is introduced. We will see that $\mathrm{Ab}$ is better than $\mathrm{Grp}$, one aspect of this is that the product and coproduct for a certain set of objects in $\mathrm{Ab}$ are the same, which is not true in $\mathrm{Grp}$, after the notion of limits and colimits are introduced.\footnote{``Life is simpler in $\mathrm{Ab}$ than in $\mathrm{Grp}$.'' -- Paolo Aluffi.}
	\item The category of rings, denoted by $\mathrm{Ring}$, in which:
		\begin{itemize}
			\item Objects: All rings.
			\item Morphisms: All ring homomorphisms.
			\item Composition law: Composition of set-functions.
		\end{itemize}
	The identity morphism is the identity ring homomorphism, and a morphism of $\mathrm{Ring}$ is an isomorphism if and only if it is a ring isomorphism. The ring $\mathbb{Z}$ of integers is the initial object in $\mathrm{Ring}$, and the trivial ring $\{0,1\}$ is the terminal object.
	\item The category of finite-dimensional vector spaces over a given field $\mathbb{F}$, denoted by $\mathrm{Vect}_\mathbb{F}$, in which:
		\begin{itemize}
			\item Objects: All finite-dimensional vector spaces over $\mathbb{F}$.
			\item Morphisms: All linear maps between these vector spaces.
			\item Composition law: Composition of set-functions.
		\end{itemize}
	The zero-dimentional space $0$ is the initial and terminal object of $\mathrm{Vect}_\mathbb{F}$. Again, $\mathrm{Vect}_\mathbb{F}$ is locally small but not small, for its collection of objects contains the collection of all one-dimensional vector spaces, and there is an injection from the collection of all one-point sets that sends every one-point set to the one-dimensional vector space generated by taking that set as its basis. The collection of all one-point sets is not a set, for there is an injection from the collection of all sets that sends every set $A$ to the one-point set $\{A\}$ whose only element is $A$.
	\item *Given a topological space $X$, then it induces a category, denoted by $\mathrm{T}X$, in which:
		\begin{itemize}
			\item Objects: All points in $X$.
			\item Morphisms: Path-homotopy classes of pathes in $X$, with their domain and codomain being the initial point and final point of the pathes, respectively.
			\item Composition law: For any $[f]$ and $[g]$ morphisms of $\mathrm{T}X$ with $\cod[f]=\dom [g]$, their composition $[gf]=[g][f]$ is the path-homotopy class of
			\[gf=\left\{
				\begin{aligned}
					&f(2s)\ \ \ \ \ \ \ \ \ \  &s\in[0,\frac{1}{2}]\\
					&g(2s-1)&s\in[\frac{1}{2},1]
				\end{aligned}
			\right.
			\]
		\end{itemize}
	The category $\mathrm{T}X$ is well-defined, c.f. Munkres, J. \textsl{Topology} $\S$51.
	\item *The category of topological spaces, denoted by $\mathrm{Top}$, in which:
		\begin{itemize}
			\item Objects: All topological spaces.
			\item Morphisms: All continuous maps between topological spaces.
			\item Composition law: Composition of set-functions.
		\end{itemize}
	A morphism of $\mathrm{Top}$ is an isomorphism if and only if it is a homeomorphism.
	\item The category of non-negative integers no more than $n$, denoted by $[\mathrm{n}]$, in which:
	 	\begin{itemize}
	 		\item Objects: Integers $0,1,\cdots,n$.
	 		\item Morphisms: For each pair of integers $(m_1,m_2)$ with $m_1\leq m_2$, an arrow pointing from $m_1$ to $m_2$. That is, the relation ``$\leq$''. $\Hom_{[\mathrm{n}]}(m_1,m_2)$ is a singleton if $m_1\leq m_2$ and is empty if $m_1>m_2$.
	 		\item Composition law: The composition follows from the transitivity of ``$\leq$''. 
	 	\end{itemize}
	Note that the relation ``$\leq$'' can't be replaced by ``<'', for the latter does not give identity morphisms. The category $[\mathrm{n}]$ is also called the \textbf{ordinal category} $\mathrm{n+1}$.
	\item The category of all non-negative integers, denoted by $\omega$, in which:
		\begin{itemize}
			\item Objects: All non-negative integers, namely all elements in $\mathbb{N}$.
			\item Morphisms: The relation ``$\leq$''.
			\item Composition law: The transitivity of ``$\leq$''.
		\end{itemize}
	This example, along with (ix), are very simple categories but will appear from time to time in our further study of category theory. In fact, a basic object for us to establish the notion of $\infty$-category is the category of all categories of non-negative integers, c.f. Example 2.2.(iii).
	\end{enumerate}
	\end{example}
We are mainly interested in $\mathrm{Set}$, $\mathrm{Grp}$ and $\mathrm{Ab}$, and we will take a fancy glance at $\mathrm{Top}$ if possible. Note that isomorphic in these categories coincident with our traditional equivalence relations between their objects, namely equinumerous between sets, isomorphic between groups, homeomorphic between topological spaces.\par
We state a few more notions about morphism before we end this section.
	\begin{definition}[Automorphism]
	Given a category $\mathcal{C}$. An \textbf{automorphism} of $\mathcal{C}$ is an endomorphism which is also an isomorphism. Given $X\in\Obj(\mathcal{C})$, we denote the collection of all automorphisms of $\mathcal{C}$ with domain $X$ by $\Aut_\mathcal{C}(X)$. The subscript could be omited, like always.
	\end{definition}
It's clear that $\Aut_\mathcal{C}(X)\subset \End_\mathcal{C}(X)$. When $\Aut_\mathcal{C}(X)$ is a set, it has the structure of a group, seen as the subcategory of $\mathcal{C}$ where the only object is $X$ and the morphisms are all elements in $\Aut_\mathcal{C}(X)$.
	\begin{definition}[Monomorphism, Epimorphism]
	A morphism $f$ of a category $\mathcal{C}$ is \\
	(i) a \textbf{monomorphism} if for any parallel morphisms $h,k$ of $\mathcal{C}$ with $\cod h=\cod k=\dom f$,
	\[fh=fk\ \ \Rightarrow\ \ h=k;\]
	(ii) an \textbf{epimorphism} if for any parallel morphisms $h,k$ of $\mathcal{C}$ with $\dom h=\dom k=\cod f$, 
	\[hf=kf\ \ \Rightarrow\ \ h=k.\]
	\end{definition}
In the category $\mathrm{Set}$, a monomorphism is exactly an injection and an epimorphism is exactly a surjection. This is also true in the category $\mathrm{Grp}$. However, this needs not hold for all categories having set-functions as its morphisms. For example, the inclusion $\mathbb{Z}\hookrightarrow \mathbb{Q}$ is not surjective, but it is an epimorphism in $\mathrm{Ring}$.
\newpage

\section{Functor}
Before we state the definition of a functor, we state an important category first, the opposite category.
	\begin{definition}[Opposite Category]
	Given a category $\mathcal{C}$. The \textbf{opposite category} of $\mathcal{C}$, denoted by $\mathcal{C}^{op}$, is a category in which:
		\begin{itemize}
			\item Objects: All objects of $\mathcal{C}$.
			\item Morphisms: All morphisms of $\mathcal{C}$ with their domain and codomain reversed. That is, given a morphism $f$ of $\mathcal{C}$, we denote its corresponding morphism in $\Mor(\mathcal{C}^{op})$ by $f^{op}$, then there is $\dom f^{op}=\cod f$ and $\cod f^{op}=\dom f$.
				\item Composition law: The induced composition law in $\mathcal{C}$. That is, for any $f^{op},g^{op}$ morphisms of $\mathcal{C}^{op}$ with $\cod f^{op}=\dom g^{op}$, their composition is given by
			\[g^{op}f^{op}=(fg)^{op},\]
		where $fg$ is the composition of $f$ and $g$ in $\mathcal{C}$. 
		\end{itemize}
	\end{definition}
	The opposite category gives us a categorical way to define the notion of opposite group:
	\begin{definition}[Opposite Group]
	Given a group $G$, seen as a category $\mathrm{B}G$. The \textbf{opposite group} of $G$ is the group $G^{op}$ whose induced category is the opposite category of $\mathrm{B}G$, namely
	\[\mathrm{B}(G^{op})=(\mathrm{B}G)^{op}.\]
	\end{definition}
	We will see soon that the opposite group is a special case of ``left action'', and that every group $G$ is ``naturally isomorphic'' to its opposite group $G^{op}$ by the group homomorphism $\varphi:G\to G^{op}:g\mapsto g^{-1}$. \par
	We now state what a functor is:
	\begin{definition}[Covariant Functor]
	Given two categories $\mathcal{C}$ and $\mathcal{D}$. A \textbf{covariant functor} $F$ from $\mathcal{C}$ to $\mathcal{D}$, denoted by $F:\mathcal{C}\to \mathcal{D}$, consists of:
	\begin{itemize}
		\item For each object $c\in \Obj(\mathcal{C})$, an object $Fc\in\Obj(\mathcal{D})$.
		\item For each morphism $f\in\Mor(\mathcal{C})$, a morphism $Ff\in\Obj(\mathcal{D})$ with $\dom Ff=F\dom f$ and $\cod Ff=F\cod f$, i.e.
		\[f:\dom f\to \cod f\ \ \ \ \mapsto\ \ \ \ Ff:F\dom f\to F\cod f.\]
	\end{itemize}
	s.t. 
	\begin{itemize}
		\item For any composable morphisms $f,g$ in $\mathcal{C}$, $(Fg)(Ff)=F(gf)$.
		\item For each object $c$ in $\mathcal{C}$, $F(1_c)=1_{Fc}$.
	\end{itemize}
	The last two conditions for a functor are called \textbf{functoriality axioms}. The category $\mathcal{C}$ is called the \textbf{domain} of $F$, and $\mathcal{D}$ is called the \textbf{codomain} of $F$.
	\end{definition}
	\begin{definition}[Contravariant Functor]
	A \textbf{contravariant functor} $F$ from $\mathcal{C}$ to $\mathcal{D}$ is a covariant functor $F:\mathcal{C}^{op}\to \mathcal{D}$. Functors have an evident way\footnote{We will soon see that natural transformations have two ways of composition.} of composition: Given functors $F:\mathcal{C}\to \mathcal{D}$ and $G:\mathcal{D}\to \mathcal{E}$, their composition $GF:\mathcal{C}\to \mathcal{E}$ is defined by
	\[GFc\coloneqq G(Fc),\ \ \forall c\in\Obj(\mathcal{C})\]
	and
	\[GFf\coloneqq G(Ff),\ \ \forall f\in\Mor(\mathcal{C}).\]
	\end{definition}
	Usually, we would omit the word ``covariant'' and say only ``functor'' in place of ``covariant functor''. To avoid unnatural arrow-theoretic representations, a morphism in the domain of a contravariant functor $F:\mathcal{C}^{op}\to \mathcal{D}$ will always be depicted as an arrow $f:c\to c'$ in $\mathcal{C}$. Graphically, the mapping on morphisms given by a contravariant functor is depicted as follows:
	\begin{center}
	\begin{tikzcd}
	\mathcal{C}\arrow[r,"F"]
	 & \mathcal{D}\\[-15pt]
	 c\arrow[d,swap,"f",""{name=A,below}] 
	 &Fc \\
	 c'
		&Fc'\arrow[u,swap,"Ff",""{name=B,below}]
		\arrow[mapsto, from=A, to=B,shorten <= 1.3em, shorten >= 1.5em]
	\end{tikzcd}
	\end{center}
\par Considering the definition of group under the categorical view, we can now redefine what a group homomorphism is:
	\begin{definition}[Group Homomorphism]
	Given two groups $G$ and $H$ seen as categories, a \textbf{group homomorphism} $f$ from $G$ to $H$ is a functor $f:G\to H$.
	\end{definition}
	We now state a lemma about functors here, whose proof is immediate:
	\begin{lemma}
	Functors preserve isomorphisms.
	\end{lemma}
	With this lemma, we can redefine and extend the notion of an action of a group:
	\begin{definition}[Action]
	Let $G$ be a group, seen as the category $\mathrm{B}G$. Given a category $\mathcal{C}$. An \textbf{action} of $G$ on an object $X\in\Obj(\mathcal{C})$ is expressed by a functor $F:\mathrm{B}G\to \mathcal{C}$, under which the image of the only object of $\mathrm{B}G$ is $X$. To be explicit, each element $g\in G$ gives by $F$ a morphism $Fg\in\End_\mathcal{C}(X)$ (In fact, $\Aut_\mathcal{C}(X)$; see Corollary 2.1.1). For any two elements $h,g\in G$, there is $(Fh)(Fg)=F(hg)$; for the identity element $e\in G$, $Fe=1_X$.\par
	When $\mathcal{C}=\mathrm{Set}$, the definition above coincidents with what we have defined to be an action of a group in the course of abstract algebra, and the object $X$ endowed with such an action is called a $G$\textbf{-set}. When $\mathcal{C}=\mathrm{Vect}_\mathbb{F}$, the object $X$ is called a $G$\textbf{-representation}. When $\mathcal{C}=\mathrm{Top}$, the object $X$ is called a $G$\textbf{-space}.\par
	The action expressed by a functor $\mathrm{B}G\to \mathcal{C}$ is sometimes called a \textbf{left action}. A \textbf{right action} is expressed by a functor $\mathrm{B}G^{op}\to \mathcal{C}$. Given a right action of a group, then it induces a left action of this group's opposite group.
	\end{definition}
	Lemma 2.1 gives immediately that
	\begin{corollary}
	When a group $G$ acts (functorially) on an object $X$ of a category $\mathcal{C}$, its elements $g$ must act by automorphisms; moreover, the inverse of the automorphism given by $g$ is the automorphism given by $g^{-1}$.
	\end{corollary}
	\begin{example}
	Here comes several examples of functors below:
	\begin{enumerate}[label=(\roman*)]
	\item Given a locally small category $\mathcal{C}$ and an object $c\in \Obj(\mathcal{C})$, then $c$ induces a functor $\Hom_{\mathcal{C}}(c,-):\mathcal{C}\to \mathrm{Set}$ and a contravariant functor $\Hom_{\mathcal{C}}(-,c):\mathcal{C}^{op}\to \mathrm{Set}$. See the diagrams below:
	
	\begin{center}
	\begin{tikzcd}
	\mathcal{C}\arrow[r,"\Hom_\mathcal{C}(c{,}-)"]
	 & \mathrm{Set}\\[-15pt]
	 x\arrow[d,swap,"f",""{name=A,below}] 
	 &\Hom_\mathcal{C}(c,x) \arrow[d,"f_*",""{name=B,below}]\\
	 y
		&\Hom_\mathcal{C}(c,y)
		\arrow[mapsto, from=A,to=B,shorten <= 1.3em, shorten >= 3em]
	\end{tikzcd}\ \ \ \ \ \ \ \ \ 
	\begin{tikzcd}
	\mathcal{C} ^{op}\arrow[r,"\Hom_\mathcal{C}(-{,}c)"]
	 & \mathrm{Set}\\[-15pt]
	 x\arrow[d,swap,"f",""{name=A,below}] 
	 &\Hom_\mathcal{C}(x,c) \\
	 y
		&\Hom_\mathcal{C}(y,c)\arrow[u,swap,"f^*",""{name=B,below}]
		\arrow[mapsto, from=A,to=B,shorten <= 1.3em, shorten >= 3em]
	\end{tikzcd}
	\end{center}
	The sign $f_*$ stands for ``composing $f$ by left''. That is, it is a set-function from $\Hom_\mathcal{C}(c,y)$ to $\Hom_\mathcal{C}(c,x)$ induced by $f$, which maps each element $g\in \Hom_\mathcal{C}(c,y)$ to $fg\in \Hom_\mathcal{C}(c,x)$. Dually, $f^*$ stands for ``composing $f$ by right''.\par
	These two functors are significantly important, called \textbf{functors represented by $c$}. We shall learn them more carefully after the notion of natural transformation is introduced.
	\item Given two categories $\mathcal{J}$ and $\mathcal{C}$. Given an object $c\in \Obj(\mathcal{C})$, then it induces a \textbf{constant functor} $c:\mathcal{J}\to \mathcal{C}$, which sends all objects of $\mathcal{J}$ to $c\in\Obj(\mathcal{C})$ and all morphisms of $\mathcal{J}$ to $1_c\in\End_\mathcal{C}(c)$. This functor is trivial but useful. We will use it to define the notion of cones, in order to introduce the notion of limits and colimits.
	\item Given a functor $F:\mathcal{C}\to \mathcal{D}$, the \textbf{opposite functor} of $F$, $F^{op}:\mathcal{C}^{op}\to \mathcal{D}^{op}$, is defined by nothing but taking everything to what $F$ brings it to:
	\[F^{op}c\coloneqq Fc,\ \ F^{op}f^{op}\coloneqq (Ff)^{op},\ \ \forall c\in\Obj(\mathcal{C}), f\in\Mor(\mathcal{C}).\]
	\item Given a category $\mathcal{C}$, then there is an identity functor $1_\mathcal{C}:\mathcal{C}\to\mathcal{C}$ that sents everything in $\mathcal{C}$ to itself.
	\item For categories with objects having underlying sets and morphisms having underlying set-functions with composition the composition of set-functions (such as $\mathrm{Grp}$, $\mathrm{Ab}$, $\mathrm{Ring}$, etc.), there is a \textbf{forgetful functor} from these categories to $\mathrm{Set}$. For example, the forgetful functor $U:\mathrm{Grp}\to \mathrm{Set}$ sends a group to its underlying set and a group homomorphism to its underlying set-function.
	\item There is a functor $(-)^*:\mathrm{Vect}_\mathbb{F}^{op}\to \mathrm{Vect}_\mathbb{F}$ that takes a vector space $V$ to its \textbf{dual vector space} $V^*\coloneqq\Hom_{\mathrm{Vect}_\mathbb{F}}(V,\mathbb{F})$. It is somehow very similar to the contravariant functor $\Hom_\mathcal{C}(-,c)$ in (i), hence we shall not explain more about it.
	\item There is a functor $F:\mathrm{Set}\to \mathrm{Grp}$ that sends a set $X$ to the \textbf{free group} on $X$. We shall return to this example after we defined what a free group is.
	\end{enumerate}
	\end{example}
	\begin{example}
	Also, with functors, we can now have more examples of categories:
		\begin{enumerate}[label=(\roman*)]
			\item The category of all small categories, denoted by $\mathrm{Cat}$, in which:
				\begin{itemize}
					\item Objects: All small categories.
					\item Morphisms: All functors between small categories.
					\item Composition: Composition of functors
				\end{itemize}
			\item The category of all locally small categories, denoted by $\mathrm{CAT}$, in which:
				\begin{itemize}
					\item Objects: All locally small categories.
					\item Morphisms: All functors between locally small categories.
					\item Composition: Composition of functors.
				\end{itemize}
				
			\item *The category of all categories of non-negative integers, also called the \textbf{simplex category}, denoted by $\triangle$, in which:
				\begin{itemize}
					\item Objects: All categories of non-negative integers. That is, $[\mathrm{n}]$ for all $n\in \mathbb{N}$.
					\item Morphisms: All functors between these categories.
					\item Composition: Composition of functors.
				\end{itemize}
		Given a category $\mathcal{C}$, then a functor $F:\triangle^{op}\to \mathcal{C}$ is called a \textbf{simplicial object} in $\mathcal{C}$. When $\mathcal{C}=\mathrm{Set}$, it is called a \textbf{simplicial set}; when $\mathcal{C}=\mathrm{Grp}$, it is called a \textbf{simplicial group}, etc.
		\end{enumerate}
	\end{example}
	It's easy to see that $\mathrm{Cat}$ is a subcategory of $\mathrm{CAT}$. The \textbf{empty category} which consists of no object is the initial object of both $\mathrm{Cat}$ and $\mathrm{CAT}$, and the ordinal category $\mathrm{1}$ is the terminal object of both. The Russell's paradox suggests that there should not be a category having itself as an object of it, hence there is no category of all categories; and this implies that $\mathrm{Cat}$ is not small and $\mathrm{CAT}$ is not locally small.\par
	We now state the notion of product of two categories, after which we can ``combine'' the two functors in Example 2.1 (i) to one functor.
	\begin{definition}[Product of two categories]
	Given two categories $\mathcal{C}$ and $\mathcal{D}$, their \textbf{product}, denoted by $\mathcal{C}\times\mathcal{D}$, is a category in which
	\begin{itemize}
		\item Objects: All ordered pairs $(c,d)$ where $c\in\Obj(\mathcal{C})$ and $d\in\Obj(\mathcal{D})$.
		\item Morphisms: All ordered pairs $(f,g):(c,d)\to (c',d')$, where $f\in\Hom_\mathcal{C}(c,c')$ and $g\in\Hom_\mathcal{D}(d,d')$.
		\item Composition: Componentwise composition according to the composition in $\mathcal{C}$ and $\mathcal{D}$ respectively.
	\end{itemize}
	\end{definition} 
	Similarly, two functors can be ``producted'' together:
	\begin{definition}[Product of two functors]
	Given two functors $F:\mathcal{C}\to \mathcal{D}$, $G:\mathcal{X}\to\mathcal{Y}$, then there is a \textbf{product functor} of them, denoted by $F\times G:\mathcal{C}\times\mathcal{X}\to\mathcal{D}\times\mathcal{Y}$, with everything defined component-wise, i.e.:
	\[F\times G(c,x)\coloneqq (Fc,Gx)\in\Obj(\mathcal{D}\times\mathcal{Y}),\ \ \forall (c,x)\in\Obj(\mathcal{C}\times\mathcal{X}),\]
	and 
	\[F\times G(f,g)\coloneqq (Ff,Gg)\in\Mor(\mathcal{D}\times\mathcal{Y}),\ \ \forall (f,g)\in\Mor(\mathcal{C}\times\mathcal{X}).\]
	\end{definition}
	We end this section by our ``combined'' functor. Note that it is not the product functor of $\Hom_\mathcal{C}(-,c)$ and $\Hom_\mathcal{C}(c,-)$. It is an important exmaple in adjunction, but we wouldn't cover that so far.
	\begin{definition}[Two-sided represented functor]
	If $\mathcal{C}$ is locally small, then there is a \textbf{two-sided represented functor} 
	\[\Hom_\mathcal{C}(-,-):\mathcal{C}^{op}\times \mathcal{C}\to \mathrm{Set},\]
	that maps an object $(x,y)\in\Obj(\mathcal{C}^{op}\times \mathcal{C})$ to the set $\Hom_\mathcal{C}(x,y)$, a morphism $(f,h)\in\Mor(\mathcal{C}^{op}\times \mathcal{C})$ to the function $(f^*,h_*)$ defined by:
	\[(f^*,h_*)(g)\coloneqq hgf,\ \ \forall g\in\Hom_\mathcal{C}(\cod f,\dom h).\]
	Note that here $f$ is seen as a morphism of $\mathcal{C}$ instead of its opposite category $\mathcal{C}^{op}$, in order to avoid unnatural notations.
	\end{definition}
\newpage
\section{Natural Transformation}
Natural transformations characterize the notion of naturality. After natural transformation is introduced, we will see an interesting example of category, the functor category. After that, we shall introduce the notion of representable functors, the character of which really shines and will lead us deeper into the category theory.
	\begin{definition}[Natural Transformation]
	Given two parallel functors $F,G:\mathcal{C}\rightrightarrows \mathcal{D}$, a \textbf{natural transformation} $\alpha$ from $F$ to $G$, denoted by $\alpha:F\Rightarrow G$, consists of:
		\begin{itemize}
			\item for each object $c\in\Obj(\mathcal{C})$, a morphism $\alpha_c\in\Hom_\mathcal{D}(Fc,Gc)$, called the \textbf{component} of $\alpha$ at $c$,
		\end{itemize}
		s.t. the following diagram
		\begin{center}
			\begin{tikzcd}
	  			F\dom f\arrow[d,swap,"Ff"] \arrow[r,"\alpha_{\dom f}"] 
	 			&G\dom f \arrow[d,"Gf"]\\
				F\cod f\arrow[r,swap,"\alpha_{\cod f}"] 
				&G\cod f
		\end{tikzcd}
		\end{center}
		\textbf{commutes}, i.e., $(Gf)(\alpha_{\dom f})=(\alpha_{\cod f})(Ff)$, for all $f\in\Mor(\mathcal{C})$.\par
		A \textbf{natural isomorphism} is a natural transformation whose every component is an isomorphism. A natural isomorphism $\alpha:F\Rightarrow G$ may be denoted as $\alpha:F\cong G$. If there is a natural isomorphism between two functors, then they are called to be \textbf{naturally isomorphic}. We will see soon that naturally isomorphic is an equivalence relation.
	\end{definition}
	\begin{example}
	Here comes some examples of natural transformations:
	\begin{enumerate}[label=(\roman*)]
		\item For any finite-dimensional vector space $V$, the evaluation map $\mathrm{ev}:V\to V^{**}$ that sends $v\in V$ to the linear function $\mathrm{ev}(v):V^*\to \mathbb{F}:\varphi\mapsto \varphi(v)$ forms a natural transformation from the idenity functor $1_{\mathrm{Vect}_\mathbb{F}}$ to the double dual functor (the composition of the dual functor $(-)^*$ with the opposite functor of itself). The reader can check this directly by checking the definition of natural transformation. In fact, it is a natural isomorphism, for $\mathrm{ev}:V\to V^{**}$ is an injective linear map and $\dim V=\dim V^{**}$. Therefore, the evaluation map tells us that $V$ and $V^{**}$ are ``naturally isomorphic'', which tends to be an equivalence relation further stronger than isomorphic.
		\item The opposite group defines a functor $(-)^{op}:\mathrm{Group}\to\mathrm{Group}$ that brings a group to its opposite group and a group homomorphism $\phi:G\to H$ to $\phi^{op}:G^{op}\to H^{op}:g\mapsto \phi(g)$; the fact that $\phi^{op}$ is a group homomorphism can be verified easily. Now that the homomorphisms $\eta_G:G\to G^{op}:g\mapsto g^{-1}$ forms a natural isomorphism from the identity functor $1_{\mathrm{Grp}}$ to $(-)^{op}$, i.e., the following diagram
		\begin{center}
			\begin{tikzcd}
	  			G\arrow[d,swap,"\phi"] \arrow[r,"\eta_G"] 
	 			&G^{op} \arrow[d,"\phi^{op}"]\\
				H\arrow[r,swap, "\eta_H"] 
				&H^{op}
		\end{tikzcd}
		\end{center}
		commutes, as one can verify.
		\item Given two parallel functors $X,Y:\mathrm{B}G\to \mathcal{C}$, each defines an action of group $G$ on $X,Y\in\Obj(\mathcal{C})$ respectively, then a natural transformation $\alpha:X\Rightarrow Y$ consists of only one morphism $\alpha:X\to Y$ of $\mathcal{C}$. This single morphism (or equivalently, the natural transformation this morphism consists) is called $G$\textbf{-equivariant}, meaning that for each $g\in G$, the diagram
		\begin{center}
			\begin{tikzcd}
	  			X\arrow[d,swap,"Xg"] \arrow[r,"\alpha"] 
	 			&Y \arrow[d,"Yg"]\\
				X\arrow[r,swap,"\alpha"] 
				&Y
		\end{tikzcd}
		\end{center}
		commutes.
	\end{enumerate}
	\end{example}
	Recall the ordinal category in Example 1.1.(ix). Consider the ordinal categories $\mathrm{1}$ and $\mathrm{2}$, there is $\Obj(\mathrm{1})=\{0\}$ and $\Obj(\mathrm{2})=\{0,1\}$, and there are two evident functors $i_0,i_1:\mathrm{1}\to\mathrm{2}$ defined by $i_0:0\mapsto 0$ and $i_1:0\mapsto 1$. We keep these notations, and here comes a characterizing of natural transformations between two functors:\footnote{It's rather weak, though.}
	\begin{lemma}
	Given two parallel functors $F,G:\mathcal{C}\rightrightarrows \mathcal{D}$. Natural transformations from $F$ to $G$ correspond bijectively to functors $H:\mathcal{C}\times \mathrm{2}\to D$ s.t. the following diagram
	\begin{center}
			\begin{tikzcd}
	  			\mathcal{C}\arrow[r,"1_\mathcal{C}\times i_0"]\arrow[rd,swap,"F"]
	  			&\mathcal{C}\times \mathrm{2}\arrow[d,"H"]
	  			&\mathcal{C}\arrow[l,swap,"1_\mathcal{C}\times i_1"]\arrow[ld,"G"]\\
	  			 &\mathcal{D}& 
		\end{tikzcd}
		\end{center}
		commutes.
	\end{lemma}
	\begin{proof}
	The proof is straightforward but interesting, thus is omitted. The reader is strongly recommended to work this out by hand.\footnote{``Dear reader: don't shy away from trying this, for it is excellent, indispensable practice. Miss this opportunity and you will forever feel unsure about such manipulations.'' -- Paolo Aluffi.}
	\end{proof}
	Natural transformations can also compose with each other, and there are two ways of compositions, called horizontal composition and vertical composition. We first state the strategy of these two compositions, and see what the explicit results of the compositions are later.
	\begin{definition}[Vertical and Horizontal Compositions]
	Given three parallel functors $F,G,H:\mathcal{C}\rightrightarrows\mathcal{D}$ and natural transformations $\alpha:F\Rightarrow G$, $\beta:G\Rightarrow H$, the \textbf{vertical composition} of $\beta$ and $\alpha $ is a natural transformation $\beta\cdot \alpha:F\Rightarrow H$. See the diagrams below:
	\begin{center}
		\begin{tikzcd}
		&F\arrow[d,Rightarrow,"\alpha"]& \\
			\mathcal{C}\arrow[rr,bend left=65]\arrow[rr]\arrow[rr,bend right=65]
			&G\arrow[d,Rightarrow,"\beta"] &\mathcal{D}\\
			&H &
		\end{tikzcd}
		$\xLongrightarrow{\text{Vertical composition}}$
		\begin{tikzcd}
		&F& \\
			\mathcal{C}\arrow[rr,bend left=50]\arrow[rr,bend right=50]
			&\Downarrow \beta\cdot\alpha &\mathcal{D}\\
			&H &
		\end{tikzcd}
	\end{center}
	Given Functors $F,G:\mathcal{C}\rightrightarrows\mathcal{D}$, $H,K:\mathcal{D}\rightrightarrows\mathcal{E}$ and natural transformations $\alpha:F\Rightarrow G$, $\beta:H\Rightarrow K$, the \textbf{horizontal composition} of $\beta$ and $\alpha$ is a natural transformation $\beta\ast\alpha:HF\Rightarrow KG$. See the diagrams below:
	\begin{center}
		\begin{tikzcd}
		&F & &H &\\
		\mathcal{C}\arrow[rr,bend left=50]\arrow[rr,bend right=50] &\Downarrow \alpha 
		&\mathcal{D} \arrow[rr,bend left=50]\arrow[rr,bend right=50]
		&\Downarrow\beta &\mathcal{E}\\
		&G & &K &
		\end{tikzcd}
		$\xLongrightarrow{\text{Horizontal composition}}$
		\begin{tikzcd}
		&HF& \\
			\mathcal{C}\arrow[rr,bend left=50]\arrow[rr,bend right=50]
			&\Downarrow \beta\ast\alpha &\mathcal{E}\\
			&KG &
		\end{tikzcd}
	\end{center}
	\end{definition}
	Let's begin with the vertical composition, with which we will be able to define the category of functors between two categories.
	\begin{lemma}[vertical composition]
	Given three parallel functors $F,G,H:\mathcal{C}\rightrightarrows \mathcal{D}$ and natural transformations $\alpha:F\Rightarrow G$, $\beta:G\Rightarrow H$, then the \textbf{vertical composition} $\beta\cdot\alpha$ of $\alpha$ and $\beta$, defined by the equation
	\[(\beta\cdot\alpha)_c\coloneqq \beta_c\alpha_c,\ \ \forall c\in\Obj(\mathcal{C})\]
	is a natural transformation from $F$ to $H$. 
	\end{lemma}
	\begin{proof}
	It follows immediately from the fact that the diagram
		\begin{center}
			\begin{tikzcd}
				Fc\arrow[r,"\alpha_c"]\arrow[d,swap,"Ff"]\arrow[rr,bend left=30,dashrightarrow,"(\beta\cdot\alpha)_c"]
					& Gc\arrow[r,"\beta_c"]\arrow[d,swap,"Gf"]
					& Hc\arrow[d,"Hf"]\\
				Fc'\arrow[r,swap,"\alpha_{c'}"]\arrow[rr,swap,bend right=30,dashrightarrow,"(\beta\cdot\alpha)_{c'}"]
					& Gc'\arrow[r,swap,"\beta_{c'}"]
					&Hc'
			\end{tikzcd}
		\end{center}
		commutes for any morphism $f:c\to c'$ of $\mathcal{C}$.
	\end{proof}
	One can learn immediately from the diagram above that if $\alpha$ and $\beta$ are both natural isomorphisms, so is $\beta\cdot \alpha$. Here comes the category of functors:
	\begin{definition}[Category of Functors]
	Given two category $\mathcal{C}$ and $\mathcal{D}$, then all functors from $\mathcal{C}$ to $\mathcal{D}$ consist a category, denoted by $\mathcal{D}^\mathcal{C}$, in which:
		\begin{itemize}
			\item Objects: All functors 	from $\mathcal{C}$ to $\mathcal{D}$.
			\item Morphisms: All natural transformations between these functors.
			\item Composition law: The vertical composition of natural transformations.
		\end{itemize}
	\end{definition}
	Here are some immediate facts about $\mathcal{D}^\mathcal{C}$: The identity morphism for an object $F\in\Obj(\mathcal{D}^\mathcal{C})$ is the identity natural transformation $1_F:F\Rightarrow F$  defined by $(1_F)_c\coloneqq 1_{Fc}$. The isomorphisms are exactly natural isomorphisms, and given a natural isomorphism $\alpha:F\Rightarrow G$, its inverse $\alpha^{-1}:G\Rightarrow F$ is defined by $(\alpha^{-1})_c\coloneqq (\alpha_c)^{-1}$. Here we conclude that naturally isomorphic is an equivalence relation. If $\mathcal{D}$ has an initial object, then the constant functor at this initial object is initial in $\mathcal{D}^\mathcal{C}$. Dually, if $\mathcal{D}$ has a terminal object, then the constant functor at this terminal object is terminal in $\mathcal{D}^\mathcal{C}$.
	\begin{remark}[sizes of functor categories]
	Care should be taken when discussing functor categories. If $\mathcal{C}$ and $\mathcal{D}$ are both small, then $\mathcal{D}^\mathcal{C}$ is again a small category. However, if both are locally small, then $\mathcal{D}^\mathcal{C}$ needs not be locally small. One sufficient condition for $\mathcal{D}^\mathcal{C}$ to be locally small is that $\mathcal{C}$ is small and $\mathcal{D}$ is locally small: Given two functors $F,G\in\Obj(\mathcal{D}^\mathcal{C})$, then
	\[\Hom_{\mathcal{D}^\mathcal{C}}(F,G)\subset \prod_{c\in\Obj(\mathcal{C})}\Hom_\mathcal{D}(Fc,Gc),\]
	since a natural transformation $\alpha:F\Rightarrow G$ may be regarded as an element $(\alpha_c)_{c\in\Obj(\mathcal{C})}\in\prod_{c\in\Obj(\mathcal{C})}\Hom_\mathcal{D}(Fc,Gc)$.
	\end{remark}
	Before entering the horizontal composition, we may want to introduce to the reader the concept of equivalences of categories. 
	\begin{definition}[Equivalence of Categories]
	Given categories $\mathcal{C}$ and $\mathcal{D}$. An \textbf{equivalence of categories} (between $\mathcal{C}$ and $\mathcal{D}$) consists of two functors $F:\mathcal{C}\to\mathcal{D}$ and $G:\mathcal{D}\to\mathcal{C}$ with natural isomorphisms $\eta:1_\mathcal{C}\cong GF$ and $\epsilon:FG\cong 1_\mathcal{D}$. If there exists an equivalence between $\mathcal{C}$ and $\mathcal{D}$, then $\mathcal{C}$ and $\mathcal{D}$ are said to be \textbf{equivalent}. 
	\end{definition}
	Equivalent between categories \textsl{is} an equivalence relation. The reflexive and symmetry are immediate, but a direct proof to the transitivity is difficult. We may want to use some properties of functors to avoid that:
	\begin{definition}[Full, Faithful and Essentially Surjective on Objects]
	A functor $F:\mathcal{C}\to \mathcal{D}$ is
		\begin{itemize}
			\item \textbf{full} if for each $x,y\in\Obj(\mathcal{C})$, the map $\Hom_\mathcal{C}(x,y)\to\Hom_\mathcal{D}(Fx,Fy):f\mapsto Ff$ is surjective;
			\item \textbf{faithful} if for each $x,y\in\Obj(\mathcal{C})$, the map $\Hom_\mathcal{C}(x,y)\to\Hom_\mathcal{D}(Fx,Fy):f\mapsto Ff$ is injective;
			\item \textbf{essentially surjective on objects} if for every object $d\in\Obj(\mathcal{D})$, there exists some $c\in\Obj(\mathcal{C})$ s.t. $d\cong Fc$.
		\end{itemize}
	\end{definition}
	One can verify easily that the composition of two full (faithful, or essentially surjective on objects) functors is again full (faithful, or essentially surjective on objects, respectively). Therefore, the theorem below yields the transitivity of equivalent between categories:
	\begin{theorem}[characterizing equivalences of categories]
		Functors in an equivalence of categories are full, faithful, and essentially surjective on objects. Conversely, any functor with these properties produces an equivalence of categories.\footnote{Here we assumed the axiom of choice.} In particular, two categories are equivalent if and only if there is a functor between them which is full, faithful, and essentially surjective on objects.
	\end{theorem}
	The proof of this theorem used many techniques of diagram chasing, and is too long to be put in this thesis, for it won't be what we mainly concern about. For a complete proof, see Theorem 1.5.9, Riehl, \textsl{Category Theory in Context}.
	\begin{remark}
		For locally small categories, one may regard that two categories are equivalent if and only if they are isomorphic in the category of locally small categories with morphisms the equivalence classes of functors modulo the equivalence relation of naturally isomorphic, i.e., the category in which:
		\begin{itemize}
			\item Objects: All locally small categories.
			\item Morphisms: All equivalence classes of functors between these categories modulo naturally isomorphic. i.e., a morphism is a collection $[F]$ consists of all functors naturally isomorphic to $F$.
			\item Composition law: Given two morphisms $[F]$ and $[G]$ with $F$ and $G$ composable, then 
			\[[G][F]\coloneqq [GF].\]
		\end{itemize}
		We shall see immediately that the composition in this category is well-defined (i.e., if $F\cong F'$ and $G\cong G'$, then $GF\cong G'F'$), after the construction of horizontal composition is given.
	\end{remark}
	Here comes the horizontal composition:
	\begin{lemma}[horizontal composition]
	Given everything in the diagram
	\begin{center}
		\begin{tikzcd}
		&F & &H &\\
		\mathcal{C}\arrow[rr,bend left=50]\arrow[rr,bend right=50] &\Downarrow \alpha 
		&\mathcal{D} \arrow[rr,bend left=50]\arrow[rr,bend right=50]
		&\Downarrow\beta &\mathcal{E}\\
		&G & &K &
		\end{tikzcd}
	\end{center}
	then the \textbf{horizontal composition} $\beta\ast\alpha$ of $\alpha$ and $\beta$, defined by the equation
	\[(\beta\ast\alpha)_c\coloneqq (\beta_{Gc})(H\alpha_c)=(K\alpha_c)(\beta_{Fc}),\]
	i.e., the diagonal of the commutative diagram
	\begin{center}
		\begin{tikzcd}
			HFc\arrow[r,"\beta_{Fc}"]\arrow[d,swap,"H\alpha_c"]\arrow[dr,dashrightarrow,"(\beta\ast\alpha)_c"]
				&KFc\arrow[d,"K\alpha_c"]\\
				HGc\arrow[r,swap,"\beta_{Gc}"]
					&KGc
		\end{tikzcd}
	\end{center}
	is a natural transformation from $HF$ to $KG$.
	\end{lemma}
	\begin{proof}
	Again, it follows immediately from the fact that the diagram
	\begin{center}
		\begin{tikzcd}
			HFc\arrow[r,"H\alpha_c"]\arrow[rr,bend left=30, dashrightarrow,"(\beta\ast\alpha)_c"]\arrow[d,swap,"HFf"]
				&HGc\arrow[r,"\beta_{Gc}"]\arrow[d,swap,"HGf"]
				&KGc\arrow[d,"KGf"]\\
			HFc'\arrow[r,swap,"H\alpha_{c'}"]\arrow[rr,bend right=30,swap,dashrightarrow,"(\beta\ast\alpha)_{c'}"]
				&HGc'\arrow[r,swap,"\beta_{Gc'}"]
				&KGc'
		\end{tikzcd}
	\end{center}
	commutes for any morphism $f:c\to c'$ of $\mathcal{C}$, as one can verify.
	\end{proof}
	Finnally, and importantly, vertical and horizontal composition are compatible: the order how the composition is done does not matter. That is, they satisfy the rule of \textbf{middle four interchange}:
	\begin{lemma}[middle four interchange]
	Given functors and natural transformations
	\begin{center}
		\begin{tikzcd}
		&F\arrow[d,Rightarrow,"\alpha"]& &J\arrow[d,Rightarrow,"\gamma"] &\\
			\mathcal{C}\arrow[rr,bend left=65]\arrow[rr]\arrow[rr,bend right=65]
			&G\arrow[d,Rightarrow,"\beta"] 
			&\mathcal{D}\arrow[rr,bend left=65]\arrow[rr]\arrow[rr,bend right=65]
			&L\arrow[d,Rightarrow,"\delta"] &\mathcal{E} \\
			&H & &L &
		\end{tikzcd}
	\end{center}
	the two natural transformations $(\delta\cdot\gamma)\ast(\beta\cdot\alpha),(\gamma\ast\alpha)\cdot(\delta\ast\beta):JF\Rightarrow LH$ are exactly the same:
	\begin{center}
		\begin{tikzcd}
		&F& &J &\\
			\mathcal{C}\arrow[rr,bend left=50]\arrow[rr,bend right=50]
			&\Downarrow \beta\cdot\alpha 
			&\mathcal{D}\arrow[rr,bend left=50]\arrow[rr,bend right=50]
			&\Downarrow\delta\cdot\gamma &\mathcal{E} \\
			&H & &L &
		\end{tikzcd}
		$=$
		\begin{tikzcd}
		&JF\arrow[d,Rightarrow,"\gamma\ast\alpha"]& \\
			\mathcal{C}\arrow[rr,bend left=55]\arrow[rr]\arrow[rr,bend right=55]
			&KG\arrow[d,Rightarrow,"\delta\ast\beta"] &\mathcal{E}\\
			&LH &
		\end{tikzcd}
	\end{center}
	\end{lemma}
	\begin{proof}
		It is nothing but the fact that the diagram
		\begin{center}
			\begin{tikzcd}
				JFc\arrow[d,"J\alpha_c"]\arrow[dd,bend right,dashrightarrow, swap,"J(\beta\cdot\alpha)_c"]\arrow[ddrr,bend left=60,dashrightarrow, "(\delta\cdot\gamma)\ast(\beta\cdot\alpha)_c"] \arrow[dr,dashrightarrow,"(\gamma\ast\alpha)_c"]& &\\
				JGc\arrow[r,"\gamma_{Gc}"]\arrow[d,"J\beta_c"]
					&KGc\arrow[d,"K\beta_c"]\arrow[dr,dashrightarrow,"(\delta\ast\beta)_c"]&\\
				JHc\arrow[r,swap,"\gamma_{Hc}"]\arrow[rr,dashrightarrow, bend right,swap,"(\delta\cdot\gamma)_{Hc}"]
					&KHc\arrow[r,swap,"\delta_{Hc}"]
					&LHc
			\end{tikzcd}
		\end{center}
		commutes for any $c\in\Obj(\mathcal{C})$, as one can verify.
	\end{proof}
\newpage
\section{Representable Functors and the Yoneda Lemma}
Remind the functors we gave in Example 2.1.(i): given $c\in\Obj(\mathcal{C})$ where $\mathcal{C}$ is locally small, then there are two functors $\Hom_\mathcal{C}(c,-):\mathcal{C}\to\mathrm{Set}$ and $\Hom_\mathcal{C}(-,c):\mathcal{C}^{op}\to \mathrm{Set}$. We called them functors represented by $c$. However, the concept of representable is able to be generalized, and these two functors are so special and so important that we have much to say even for the generalized concept. In this seciton we give the definition of representable functors and 	raise a few questions to be solved in the next section. We will give an introduction to their solutions in this section, though: we will introduce the Yoneda lemma.
	\begin{definition}[Representable Functors]
		A covariant (or contravariant) functor $F$ from a locally small category $\mathcal{C}$ to $\mathrm{Set}$ is said to be \textbf{representable} if there exists an object $c\in\Obj(\mathcal{C})$ s.t. there is a natural isomorphism between $F$ and the functor $\Hom_\mathcal{C}(c,-)$ (or $\Hom_\mathcal{C}(-,c)$, respectively), in which case we say that the functor $F$ is \textbf{represented by} the object $c$. A \textbf{representation} for a representable covariant (or contravariant) functor $F$ is a choice of $c\in\Obj(\mathcal{C})$ together with a specified natural isomorphism $\Hom_\mathcal{C}(c,-)\cong F$ (or $\Hom_\mathcal{C}(-,c)\cong F$, respectively).
	\end{definition}
	Note that the domain $\mathcal{C}$ of a representable functor is required to be locally small, so that the hom-functors $\Hom_\mathcal{C}(c,-)$ and $\Hom_\mathcal{C}(-,c)$ do send objects of $\mathcal{C}$ to sets.
	\begin{example}
	Here comes some examples of representable functors which we have already got familiar with.
		\begin{enumerate}[label=(\roman*)]
			\item The identity functor $1_\mathrm{Set}:\mathrm{Set}\to\mathrm{Set}$ is represented by the singleton set $1$. The natural isomorphism $\Hom_\mathrm{Set}(1,-)\cong 1_\mathrm{Set}$ consists of maps $\Hom_\mathrm{Set}(1,X)\to X$ that maps each element in $\Hom_\mathrm{Set}(1,X)$ to the unique element in its image in $X$. One can verify easily that the diagram 
			\begin{center}
			\begin{tikzcd}
	  			\Hom_\mathrm{Set}(1,X)\arrow[d,swap,"f_*"] \arrow[r,"\cong"] 
	 			&X \arrow[d,"f"]\\
				\Hom_\mathrm{Set}(1,Y)\arrow[r,swap,"\cong"] 
				&Y
			\end{tikzcd}
			\end{center}
		commutes. For this reason, one may denote an element in $\Hom_\mathrm{Set}(1,X)$ by its image, i.e., $x\in X$ may also denote the function $x:1\to X:1\mapsto x$.
		\item The forgetful functor $U:\mathrm{Grp}\to \mathrm{Set}$ is represented by the additive group $\mathbb{Z}$. The natural isomorphism $\Hom_\mathrm{Grp}(\mathbb{Z},-)\cong U$ consists of bijections $\Hom_\mathrm{Grp}(\mathbb{Z},G)\to UG:\varphi\mapsto \varphi(1)$. The bijectivity follows from that a group homomorphism from $\mathbb{Z}$ to $G$ is determined by the image of $1$ the generator of $\mathbb{Z}$. We will see that $\mathbb{Z}$ is the \textbf{free group on a single generator} after we define free groups.
		\item *For a generalization to (ii): Given an index set $J$. The functor $U(-)_J:\mathrm{Grp}\to\mathrm{Set}$ that sends a group $G$ to the set of $J$-tuples of elements of $G$ is represented by the \textbf{free group $F(J)$ on $J$}. Similarly, the functor $U(-)_J:\mathrm{Ab}\to \mathrm{Set}$ is represented by the \textbf{free abelian group $\bigoplus_{j\in J}\mathbb{Z}_j$ on $J$}.\footnote{$\bigoplus_{j\in J}\mathbb{Z}_j$ is denoted as $\mathbb{Z}^{\oplus J}$ in \textsl{Algebra: Chapter 0}}
		\end{enumerate}
	\end{example}
	As the notion of representation is given, here raises a few questions:
	\begin{itemize}
		\item If two objects represent a same functor, are they isomorphic?
		\item What data is involved in the construction of a natural isomorphism in the representation of a functor $F$?
		\item *To reader who has known something about universal properties: we assert that the universal property of an object in a locally samll category can be expressed by representable functors. How do the universal property expressed by functors relate to initial and terminal objects?
	\end{itemize}
	The answer to the first question is ``yes''. One may prove it by hand right now, but we shall not put the proof here; it happens to be an immediate result of Yoneda lemma. The Yoneda lemma also provides insights for the other two questions. In fact, the Yoneda lemma is arguably the most important result in category theory:\footnote{``... although it takes some time to explore the depths of the consequences of this simple statement.'' -- Emily Riehl.}
	\begin{theorem}(Yoneda lemma)
		For any functor $F:\mathcal{C}\to \mathrm{Set}$ whose domain $\mathcal{C}$ is locally small, for any object $c\in\Obj(\mathcal{C})$, the function $\Phi:\Hom_{\mathrm{Set}^{\mathcal{C}}}(\Hom_\mathcal{C}(c,-),F)\to Fc:\alpha\mapsto \alpha_c(1_c)$ is a bijection, concluding that
		\[\Hom_{\mathrm{Set}^\mathcal{C}}(\Hom_\mathcal{C}(c,-),F)\cong Fc,\]
		hence $\Hom_{\mathrm{Set}^\mathcal{C}}(\Hom_\mathcal{C}(c,-),F)$ is a set. Moreover, $\Phi$ is natural with respect to both $c$ and $F$, i.e., seen as the component of a natural isomorphism at the object $(c,F)\in\Obj(\mathcal{C}\times\mathrm{Set}^\mathcal{C})$, it consists a natural isomorphism between bifunctors:
		\[\Hom_{\mathrm{Set}^\mathcal{C}}(\Hom_\mathcal{C}(\circ,-),\diamond):\mathcal{C}\times\mathrm{Set}^\mathcal{C}\to \mathrm{Set}:(c,F)\mapsto \Hom_{\mathrm{Set}^{\mathcal{C}}}(\Hom_\mathcal{C}(c,-),F)\]
		and
		\[ev:\mathcal{C}\times\mathrm{Set}^\mathcal{C}\to \mathrm{Set}:(c,F)\mapsto Fc.\]
	\end{theorem}
	\begin{proof}
		The main point is to construct the inverse $\Psi:Fc\to\Hom_\mathcal{\mathrm{Set}^\mathcal{C}}(\Hom_\mathcal{C}(c,-),F)$ of $\Phi$. Elements in the codomain of $\Psi$ are natural transformations from $\Hom_\mathcal{C}(c,-)$ to $F$, hence given $x\in Fc$, we proceed by defining each component of the natural transformation $\Psi(x):\Hom_\mathcal{C}(c,-)\Rightarrow F$. In order to make $\Psi$ the inverse of $\Phi$, there must be $\Psi(x)_c(1_c)=\Phi(\Psi(x))=x$. For any $d\in\Obj(\mathcal{C})$, if $\Hom_\mathcal{C}(c,d)=\varnothing$, then we simply let $\Psi(x)_d$ be the empty function; if $\Hom_\mathcal{C}(c,d)$ is non-empty, then for any $f\in\Hom_\mathcal{C}(c,d)$, the diagram
		\begin{center}
			\begin{tikzcd}
	  			\Hom_\mathcal{C}(c,c)\arrow[d,swap,"f_*"] \arrow[r,"\Psi(x)_c"] 
	 			&Fc \arrow[d,"Ff"]\\
				\Hom_\mathcal{C}(c,d)\arrow[r,swap,"\Psi(x)_d"] 
				&Fd
			\end{tikzcd}
			\end{center}
			must commute. In particular, $\Psi(x)_d(f)=\Psi(x)_d(f_*(1_c))=Ff\circ\Psi(x)_c(1_c)=Ff(x)$. Therefore, we have defined $\Psi$:
			\[\Psi:Fc\to\Hom_{\mathrm{Set}^\mathcal{C}}(\Hom_\mathcal{C}(c,-),F)\ \ \ \ \ \Psi(x)_d(f)\coloneqq Ff(x),\ \ \forall f\in\Hom_\mathcal{C}(c,d),\ \ \forall d\in\Obj(\mathcal{C}),\ \ \forall x\in Fc.\]
			It remains to verify that $\Psi(x)$ is natural, and that $\Psi\circ\Phi(\alpha)=\alpha$. Both are straightforward, hence are left to the reader; note that a natural transformation from $\Hom_\mathcal{C}(c,-)$ to $F$ is determined by the image of $1_c$ of its component at $c$, as we have shown in the construction of $\Psi$.\par
			The naturality of $\Phi$ is again nothing but to check the diagram
			\begin{center}
			\begin{tikzcd}
	  			\Hom_{\mathrm{Set}^\mathcal{C}}(\Hom_\mathcal{C}(c,-),F)\arrow[d,swap,"(f^{**}{,}\alpha_*)"] \arrow[r,"\Phi_{(c,F)}"] 
	 			&Fc \arrow[d,"(\alpha_d)(Ff)=(Gf)(\alpha_c)"]\\
			\Hom_{\mathrm{Set}^\mathcal{C}}(\Hom_\mathcal{C}(d,-),G)\arrow[r,swap,"\Phi_{(d,G)}"] 
				&Gd
			\end{tikzcd}
			\end{center}
			commutes, hence is left to the reader.\footnote{For a complete proof of the Yoneda lemma, c.f. Emily Riehl, \textsl{Category Theory in Context}, Theorem 2.2.4.}
	\end{proof}
	There is a dual version of the Yoneda lemma, in which the functor $F$ is a contravariant functor from $\mathcal{C}$ to $\mathrm{Set}$, but there is no need to list it alone: it can be accessed immediately by applying the original Yoneda lemma to the functor $F:\mathcal{C}^{op}\to \mathrm{Set}$ (seen as a covariant functor with domain $\mathcal{C}^{op}$), and using the fact that $\Hom_{\mathcal{C}^{op}}(c,-)=\Hom_\mathcal{C}(-,c)$. \par
	To emphasis, the Yoneda lemma tells us that there are only a set's worth of natural transformations between $F$ and $\Hom_\mathcal{C}(c,-)$. An immediate application of the Yoneda lemma gives the Yoneda embeddings:
	\begin{corollary}[Yoneda embedding]
		The functors
		\begin{center}
	\begin{tikzcd}
	\mathcal{C}\arrow[r,hook,"y"]
	 & \mathrm{Set}^{\mathcal{C}^{op}}\\[-15pt]
	 c\arrow[d,swap,"f",""{name=A,below}] 
	 &\Hom_\mathcal{C}(-,c) \arrow[d,"f_*",""{name=B,below}]\\
	 d
		&\Hom_\mathcal{C}(-,d)
		\arrow[mapsto, from=A,to=B,shorten <= 1.3em, shorten >= 3em]
	\end{tikzcd}\ \ \ \ \ \ \ \ \ 
	\begin{tikzcd}
	\mathcal{C} ^{op}\arrow[r,hook,"y"]
	 & \mathrm{Set}^\mathcal{C}\\[-15pt]
	 c\arrow[d,swap,"f",""{name=A,below}] 
	 &\Hom_\mathcal{C}(c,-) \\
	 d
		&\Hom_\mathcal{C}(d,-)\arrow[u,swap,"f^*",""{name=B,below}]
		\arrow[mapsto, from=A,to=B,shorten <= 1.3em, shorten >= 3em]
	\end{tikzcd}
	\end{center}
	are both full and faithful. They are called the covariant and contravariant \textbf{Yoneda embeddings}. 
	\end{corollary}
	As a simple application of the Yoneda embeddings, we may reprove the Cayley's theorem:
	\begin{corollary}[Cayley's Theorem]
	 Any group is isomorphic to a subgroup of a permutation group.
	\end{corollary}
	\begin{proof}
		Given a group $G$, regard it as the one-object category $\mathrm{B}G$. The covariant Yoneda embedding tells us that the embedding
		\begin{center}
	\begin{tikzcd}
	\mathrm{B}G\arrow[r,hook,"y"]
	 & \mathrm{Set}^{\mathrm{B}G^{op}}\\[-15pt]
	 \bullet\arrow[d,swap,"g",""{name=A,below}] 
	 &\Hom_{\mathrm{B}G}(-,\bullet) \arrow[d,"g_*",""{name=B,below}]\\
	 \bullet
		&\Hom_{\mathrm{B}G}(-,\bullet)
		\arrow[mapsto, from=A,to=B,shorten <= 1.3em, shorten >= 3em]
	\end{tikzcd}
	\end{center}
	is full and faithful. Note that the forgetful functor $U:\mathrm{Set}^{\mathrm{B}G^{op}}\to \mathrm{Set}:F\mapsto F\bullet$ is also faithful. Since $\Hom_{\mathrm{B}G}(\bullet,\bullet)=G$, the composition $Uy$ yields that any $G$-equivariant\footnote{c.f. Example 3.1.(iii)} endomorphism on the $G$-set $G$ is an automorphism in the category $\mathrm{Set}$, and that the set $G_{eq}$ of $G$-equivariant endomorphisms on the $G$-set $G$ endowed with the binary operation of composition of set-functions is a group isomorphic to $G$. The fact that $G_{eq}$ is a subgroup of $\Aut_\mathrm{Set}(G)=\Sym(G)$ finishes our proof.
	\end{proof}
\newpage
\section{Universal Property}
In this section we introduce the universal property. We shall see that the idea of universal property has already played an important role (which we haven't discovered, though) in our previous study of abstract algebra, and it can not (and should not) be avoid if we want to proceed further. We will focus only on the universal properties of objects in locally small categories, since non-locally-small categories tend to be rare in many other mathematics. Also, for locally small categories, we have a very fancy definition for the universal property. However, to begin with, we will still introduce the general definition of universal property to the reader. It tends to be ambiguous to people who never saw it before, though:
\begin{definition}[Universal Property]
A certain mathematical object (or construction) is said to satisfy (or have) a \textbf{universal property}, if it could be seen as (a part of) an initial or terminal object of some other category. In particular, an initial or terminal object automatically satisfies an evident universal property. 
\end{definition}
Here by a mathematical object we mean things to be operated, such as numbers, sets, groups, etc. An explicit definition for mathematical objects is the job for philosophy, hence we shall not be bothered with that. Also, by ``could be seen'' we usually leave the object unchanged but attatch some constructions on it and generalize the whole to be objects of a category, with morphisms induced from morphisms in the category where the original object is taken from. When we say something satisfies a universal property, the context should be clear enough for the reader to figure out in what kind of category is the object universal. Now we give some examples for expressing universal properties. 
\begin{example}
	\begin{enumerate}[label=(\roman*)]
		\item Given a group homomorphism $\varphi:G\to H$, its kernel is universal respect to the following property: for any group homomorphism $g:K\to G$ s.t. $\varphi\circ g=0$, there exists a unique group homomorphism $\tilde g:K\to \ker \varphi$ s.t. $i\circ\tilde g=g$, where $i:\ker\varphi\to G$ is the inclusion. In other words, $\ker \varphi$ could be seen as the terminal object in the category where
			\begin{itemize}
				\item Objects: Group homomorphisms $g:K\to G$ s.t. $\varphi\circ g=0$;
				\item Morphisms: Given two objects $f:W\to G$, $g:K\to G$, a morphism from $f$ to $g$ is a group homomorphism $h:W\to K$ s.t. $g\circ h=f$, i.e., the diagram
				\begin{tikzcd}
				W\arrow[r,"f"]\arrow[dr,swap,"h"]&G\\
				&K\arrow[u,swap,"g"]
				\end{tikzcd}
				commutes.
			\end{itemize}
		\item Given a group $G$, then its commutation $\tilde G\coloneqq G/[xyx^{-1}y^{-1}:x,y\in G]$, where $[xyx^{-1}y^{-1}:x,y\in G]$ stands for the least normal group containing $\{xyx^{-1}y^{-1}:x,y\in G\}$ (in this case it is exactly the set $\{xyx^{-1}y^{-1}:x,y\in G\}$), is universal respect to the following property: for any group homomorphism $\varphi:G\to A$ where $A$ is abelian, there exists a unique group homomorphism $\tilde \varphi:\tilde G\to A$ s.t. $\tilde\varphi\circ\pi=\varphi$, where $\pi:G\to \tilde G$ is the quotient map. In other words, $\tilde G$ could be seen as the initial object in the category where
			\begin{itemize}
				\item Objects: Group homomorphisms $g:G\to A$ where $A$ is abelian;
				\item Morphisms: Given two objects $f:G\to H$, $g:G\to A$, a morphism from $f$ to $g$ is a group homomorphism $h:H\to A$ s.t. $h\circ f=g$, i.e., the diagram
				\begin{tikzcd}
				G\arrow[r,"g"]\arrow[d,swap,"f"]&A\\
				H\arrow[ur,swap,"h"]
				\end{tikzcd}
				commutes.
				\end{itemize}
		\item Given two sets $A$ and $B$, then their cartesian product $A\times B$ is universal with respect to the following property: for any two set-functions $f:C\to A$, $g:C\to B$, there exists a unique function $h:C\to A\times B$ s.t. $\pi_A\circ h=f$ and $\pi_B\circ h=g$, where $\pi_A:A\times B\to A:(a,b)\mapsto a$, $\pi_B:A\times B\to B:(a,b)\mapsto b$ are called the projection maps. In other words, $A\times B$ could be seen as the terminal object in the category where
			\begin{itemize}
				\item Objects: Triples $(f,g,C)$, where $f:C\to A$, $g:C\to B$ are set functions, $C$ stands for an arbitrary set;
				\item Morphisms: Given two objects $(f_1,g_1,C_1)$, $(f_2,g_2,C_2)$, a morphism from $(f_1,g_1)$ to $(f_2,g_2)$ is a set function $h:C_1\to C_2$ s.t. $f_2\circ h=f_1$, $g_2\circ h=g_1$, i.e. the diagram
				\begin{tikzcd}
				&&A\\
				C_1\arrow[urr,bend left=20,"f_1"]\arrow[drr,bend right=20,swap,"g_1"]\arrow[r,"h"]&C_2\arrow[ur,"f_2"]\arrow[dr,swap,"g_2"]&\\
				&&B
				\end{tikzcd}
				commutes.
			\end{itemize}
	The cartesian product of sets is a special case of product. The product is a very important universal property, which we shall explore in the next section.
	\end{enumerate}
\end{example}
There are many other familiar examples, such as the kernel of linear maps, the product of groups, the product topology, etc. All of which, along with those have been listed above, belong to a kind of special universal property, called limits and colimits. Hence we pause here, and leave a further discussion to the next section. We have already known that initial (or terminal) objects in a certain category are all isomorphic canonically, hence we can use the universal property of an object to re-define the object itself, which gives a definition up to canonical isomorphic. In fact, that is exactly what we are doing most of the time, even without knowing an object priorily. Note that we need to verify the existence of such object if we define something new using universal property. 
\begin{example}Here comes some examples for defining things using universal property.
	\begin{enumerate}[label=(\roman*)]
		\item Given a group homomorphism $\varphi:G\to H$. The \textbf{cokernel} of $\varphi$, denoted by $\coker \varphi$, is the codomain of the initial object in the category where
			\begin{itemize}
				\item Objects: All group homomorphisms $f:H\to K$ s.t. $f\circ \varphi=0$;
				\item Morphisms: Given two objects $f_1:H\to K_1$, $f_2:H\to K_2$, a morphism from $f_1$ to $f_2$ is a group homomorphism $h:K_1\to K_2$ s.t. $h\circ f_1=f_2$, i.e., the diagram 
				\begin{tikzcd}
				H\arrow[r,"f_1"]\arrow[dr,swap,"f_2"]&K_1\arrow[d,"h"]\\
				&K_2
				\end{tikzcd}
			commutes.
			\end{itemize}
		The reader may verify that $H/[\Ima\varphi]$ is the cokernel of $\varphi$, using some abstract algebra. 
		\item Given a set $A$. The \textbf{free group} on $A$, denoted by $F(A)$, is the codomain of the initial object in the category where
			\begin{itemize}
				\item Objects: All set-functions $f:A\to G$, where $G$ is a group;
				\item Morphisms: Given two objects $f:A\to G$ and $g:A\to H$, a morphism from $f$ to $g$ is a group homomorphism $h:G\to H$ s.t. $h\circ f=g$, i.e., the diagram
				\begin{tikzcd}
				G\arrow[r,"h"]&H\\
				A\arrow[u,"f"]\arrow[ur,swap,"g"]
				\end{tikzcd}
				commutes.
			\end{itemize}
		\item Given a set $A$. The \textbf{free abelian group} on $A$, denoted by $F^{ab}(A)$, is the codomain of the initial object in the category where
			\begin{itemize}
				\item Objects: All set-functions $f:A\to G$, where $G$ is an abelian group;
				\item Morphisms: Given two objects $f:A\to G$ and $g:A\to H$, a morphism from $f$ to $g$ is a group homomorphism $h:G\to H$ s.t. $h\circ f=g$, i.e., the diagram
				\begin{tikzcd}
				G\arrow[r,"h"]&H\\
				A\arrow[u,"f"]\arrow[ur,swap,"g"]
				\end{tikzcd}
				commutes.
			\end{itemize}

	\end{enumerate}
	For an explicit construction for free groups and free abelian groups, see either $\S$5, Chapter II, \textsl{Algebra: Chapter 0} or $\S$67 and $\S$69, Chapter 11, \textsl{Topology}. One will see that $F^{ab}(A)\cong \oplus_{j\in A}\mathbb{Z}_j$, where the latter stands for the direct sum of $\mathbb{Z}$ (see $\S$67, \textsl{Topology}, or see the next section), and is also denoted by $\mathbb{Z}^{\oplus A}$, hence we may refer to the free abelian group on $A$ simply by $\mathbb{Z}^{\oplus A}$.
\end{example}

\begin{remark}
Free groups provide us a new way to deal with groups: every group can be seen as (up to isomorphic) a free group modulo a normal subgroup. If $G\cong F(A)/R$, then $F(A)/R$ along with the isomorphism is called a \textbf{presentation} of group $G$. A group might have a number of very different presentations, while there must be at least one presentation for a group: the free group on the underlying set of group $G$, by the first isomorphism theorem, will do. See the diagram below.
\begin{center}
	\begin{tikzcd}
		F(G)\arrow[r,dashed,two heads,"\exists!"]&G\\
		G\arrow[u,"j"]\arrow[ur,swap,"id_G"]
	\end{tikzcd}
\end{center}
\end{remark}
Now we may be ready for our fancy definition of universal property on locally small categories. Recall the bijective relation $\Hom_{\mathrm{Set}^{\mathcal{C}}}(\Hom_\mathcal{C}(c,-),F)\cong Fc$ in the Yoneda lemma:
\begin{definition}[Universal Property]
	A \textbf{universal property} of an object $c\in \Obj(\mathcal{C})$ is (expressed by) a functor $F$ represented by $c$. More explicitly, a universal property of an object $c\in \mathcal{C}$ is a pair $(F,x)$ where $F$ is a  representable functor and $x\in Fc$ gives a natural isomorphism $\Hom_\mathcal{C}(c,-)\cong F$ or $\Hom_\mathcal{C}(-,c)\cong F$ via the Yoneda lemma.
\end{definition}
In particular, $\Hom_\mathcal{C}(c,-)$ (or $\Hom_\mathcal{C}(-,c)$) is a universal property of $c\in \Obj(\mathcal{C})$ (be aware of the choice of which category $\mathcal{C}$ the object $c$ is in), and the reader should have no difficulty translating the general definition to this fancy one; examples will be given right away. To translate this fancy definition to the general one, that is, to find a category and an initial or terminal object out from a representable functor (along with its representation), we need to establish a special kind of category, called \textbf{the category of elements}. 
\begin{example} We first gives some example of translations by pointing out the representable functor and its representation; the details are left to the reader.
	\begin{enumerate}[label=(\roman*)]
		\item Given a group homomorphism $\varphi:G\to H$, the universal property of its kernel is expressed by the functor $F:\mathrm{Grp}\to \mathrm{Set}:K\mapsto \{f\in\Hom_\mathrm{Grp}(K,G):\varphi\circ f=0\}$, which is represented by $\Hom_\mathrm{Grp}(-,\ker\varphi)$. The universal element is the inclusion map $j:\ker \varphi\hookrightarrow G$. 
		\item Given a group $G$, then the universal property of its commutation $\tilde G$ is expressed by the functor $F:\mathrm{Ab}\to \mathrm{Set}:H\mapsto \Hom_\mathrm{Grp}(G,H)$, which is represented by $\Hom_\mathrm{Ab}(\tilde G,-)$. The universal element is the quotient map $\pi:G\to \tilde G$.
		\item Given two sets $A$ and $B$, the universal property of their product $A\times B$ is expressed by the functor $F:\mathrm{Set}\to \mathrm{Set}:S\mapsto \{(f,g):f\in\Hom_\mathrm{Set}(S,A),g\in\Hom_\mathrm{Set}(S,B)\}$, which is represented by the $\Hom_\mathrm{Set}(-,A\times B)$. The universal element is the pair of projection maps $(\pi_A,\pi_B)$.
	\end{enumerate}
\end{example}
\begin{definition}[Category of Elements]
(Covariant) The \textbf{category of elements} of a covariant functor $F:\mathcal{C}\to \mathrm{Set}$, denoted by $\int F$, consists of 
	\begin{itemize}
		\item Objects: All pairs $(c,x)$ where $c\in\Obj(\mathcal{C})$ and $x\in Fc$;
		\item Morphisms: Given two objects $(c,x)$ and $(c',x')$, a morphism $(c,x)\to (c',x')$ is a morphism $f:c\to c'$ of $\mathcal{C}$ s.t. $Ff(x)=x'$.	
	\end{itemize}
(Contravariant) The \textbf{category of elements} of a contravariant functor $F:\mathcal{C}^{op}\to\mathrm{Set}$, denoted by $\int F$, consists of
	\begin{itemize}
		\item Objects: All pairs $(c,x)$ where $c\in \Obj(\mathcal{C})$ and $x\in Fc$;
		\item Morphisms: Given two objects $(c,x)$ and $(c',x')$, a morphism $(c,x)\to (c',x')$ is a morphism $f:c\to c'$ of $\mathcal{C}$ s.t. $Ff(x')=x$.
	\end{itemize}
\end{definition}
The proposition below ends our translation question:
\begin{proposition}A covariant set-valued (i.e., its codomain is $\mathrm{Set}$) functor is representable if and only if its category of elements has an initial object. Dually, a contravariant set-valued functor is representable if and only if its category of elements has a terminal object. Explicitly, the representation of a functor is initial (or terminal) in its category of elements.
\end{proposition}
\begin{proof}
By duallity, we only prove the case where the functor $F:\mathcal{C}\to \mathrm{Set}$ is covariant. The necessity is easy to see: given representation $\alpha:\Hom_\mathcal{C}(c,-)\cong F$, then for any $(d,x)\in \Obj(\int F)$, the diagram 
	\begin{center}
		\begin{tikzcd}
			\Hom_\mathcal{C}(c,c)\arrow[d,"f"]\arrow[r,"\alpha_c"]&Fc\arrow[d,"Ff"]\\
			\Hom_\mathcal{C}(c,d)\arrow[r,"\alpha_d"]&Fd
		\end{tikzcd}
	\end{center}
	commutes. We assert that $(c,\alpha_c(1_c))$ is initial. It suffices to show that there exists a unique $f:c\to d$ s.t. $Ff(\alpha_c(1_c))=x$. By the commutativity, $Ff(\alpha_c(1_c))=\alpha_d(f)$, and we are done by the bijectivity of $\alpha_d$.\\
	Now given an initial object $(c,x)$ of $\int F$, we assert that the natural transformation $\alpha:\Hom_\mathcal{C}(c,-)\Rightarrow F$ given by $x$ via the Yoneda lemma is a natural isomorphism. In particular, $\alpha_c(1_c)=x$. For any $c'\in\Obj(\mathcal{C})$, we show that $\alpha_{c'}$ is a bijection. Again, the diagram
	\begin{center}
		\begin{tikzcd}
			\Hom_\mathcal{C}(c,c)\arrow[d,"f"]\arrow[r,"\alpha_c"]&Fc\arrow[d,"Ff"]\\
			\Hom_\mathcal{C}(c,c')\arrow[r,"\alpha_{c'}"]&Fc'
		\end{tikzcd}
	\end{center}
	commutes. For any $x'\in Fc'$, since $(c,\alpha_c(1_c))$ is initial, there exists a unique $f:c\to c'$ s.t. $Ff(\alpha_c(1_c))=x'$. Since $Ff(\alpha_c(1_c))=\alpha_{c'}(f)$, the existence of such $f$ implies that $\alpha_{c'}$ is surjective, and the uniqueness implies that $\alpha_{c'}$ is injective.
\end{proof}
Before entering the next section, the reader is suggested to raise some familiar examples of universal property by hand, verify both the general definition and the fancy definition and write out the category of elements, comparing it with the category in the general definition.
\newpage
\section{Limits and Colimits}
To be finished.
\end{document}